\documentclass[10pt]{article}
\usepackage{amsmath,amssymb,hyperref,color,graphicx,fancyhdr,geometry,comment,pdftexcmds}

% The following script will result in the creation of 2 documents:
% QuizN_M_Student.pdf and QuizN_M_Key.pdf, where N = \QuizNumber and
% M = \SectionNumber. To execute it, you have to enable pdflatex to
% "escape to shell" by compiling with the following command:
%   pdflatex --shell-escape quiz_body.tex
\makeatletter
\ifcase\pdf@shellescape
  \gdef\condition{0}\or
  \ifx\condition\undefined
    \immediate\write18{
    pdflatex --jobname=\jobname
    "\gdef\string\condition{0}
    \string\input\space\jobname"}
    \immediate\write18{
    pdflatex --jobname=\jobname_Key
    "\gdef\string\condition{1}
    \string\input\space\jobname"}
    \expandafter\stop
  \fi\or
  \gdef\condition{0}\fi
\makeatother

\ifcase\condition
  % \condition = 0, student version
  \excludecomment{key}
  \or % \condition = 1, key version
  \includecomment{key}
\fi
%%%%%%%%%%%%%%%%%%%%%%%%%%%%%%%%%%%%%%%%%%%%%%%%%%%%%%%%%%%%%%%%%%%%%%%%%%%%%%%%
\setlength{\headheight}{15.2pt}
\pagestyle{fancy}

\lhead{\CourseNumber:\SectionNumber}
\chead{\textit{\CourseTitle}}
\rhead{\QuizDate}

\begin{key}
\lfoot{\textbf{KEY:} \Key}
\end{key}

\def\vs{\vspace{.85\baselineskip}}
\def\svs{\vspace{.5\baselineskip}}
\def\nvs{\vspace{-.85\baselineskip}}
\def\nsvs{\vspace{-.5\baselineskip}}
\def\vf{\vfill}
\def\ds{\displaystyle}

%%%%%%%%%%%%%%%%%%%%%%%%%%%%%%%%%%%%%%%%%%%%%%%%%%%%%%%%%%%%%%%%%%%%%%%%%%%%%%
%% FONTS, MICROTYPE, AND SPACING
%https://tex.stackexchange.com/questions/82001/microtype-settings-for-dummies
%http://www.khirevich.com/latex/microtype/
\usepackage[activate={true,nocompatibility},final,tracking=true,kerning=true,spacing=true, factor=1100,stretch=10,shrink=10]{microtype}
\usepackage[T1]{fontenc}

%Prevents odd spacing message
\microtypecontext{spacing=nonfrench}

\SetProtrusion{encoding={*},family={*}, series={*},size={6,7}}
              {1={ ,750},2={ ,500},3={ ,500},4={ ,500},5={ ,500},
               6={ ,500},7={ ,600},8={ ,500},9={ ,500},0={ ,500}}

\SetExtraKerning[unit=space]
    {encoding={*}, family={*}, series={*}, size={footnotesize,small,normalsize}}
    {\textendash={400,400}, % en-dash, add more space around it
     "28={ ,150}, % left bracket, add space from right
     "29={150, }, % right bracket, add space from left
     \textquotedblleft={ ,150}, % left quotation mark, space from right
     \textquotedblright={150, }} % right quotation mark, space from left

\SetExtraKerning[unit=space]
   {encoding={*}, family={*}, series={b}, size={large,Large}}
   {1={-200,-200},
    \textendash={400,400}}
\SetTracking{encoding={*}, shape=sc}{40}

%Font Families -- https://math.ucsd.edu/~msharpe/RcntFnts.pdf
\usepackage{charter}
\usepackage[varqu,varl]{zi4}% inconsolata
\usepackage[charter,vvarbb,scaled=1.05]{newtxmath}
\usepackage[supscaled=1.2, raised=-.13em]{superiors}
%\usepackage[cal=boondoxo]{mathalfa} % mathcal
%\useosf % osf for text, not math

% Allows fonts to be rescaled arbitrarily without error messages
\usepackage{type1ec}

%%%%%%%%%%%%%%%%%%%%%%%%%%%%%%%%%%%%%%%%%%%%%%%%%%%%%%%%%%%%%%%%%%%%%%%%%%%%%%
%% COUNTERS

% problem counter
\newcounter{probcounter}
\setcounter{probcounter}{0}

% subproblem counter
\newcounter{subprobcounter}
\setcounter{subprobcounter}{0}
\renewcommand{\thesubprobcounter}{\alph{subprobcounter}}

%%%%%%%%%%%%%%%%%%%%%%%%%%%%%%%%%%%%%%%%%%%%%%%%%%%%%%%%%%%%%%%%%%%%%%%%%%%%%%
%% PROBLEM FORMATTING

% problem environment
\newcommand{\prob}[1]
	{\phantom{x}\hspace{-.25in}
	\setcounter{subprobcounter}{0}
	\refstepcounter{probcounter}
	{\large \bf \theprobcounter.} $\;$
	\parbox[t]{5.65in}{#1} \\[1pc]}

% subproblem environment
\newcommand{\subprob}[1]
	{\refstepcounter{subprobcounter}
	\hspace{.25in}
	{\bf \Roman{subprobcounter})} $\;$
	\parbox[t]{5.35in}{#1}
	\svs \\}

% unlabeled subitem environment
\newcommand{\thing}[1]
	{\hspace{.25in}
	\parbox[t]{5.35in}{#1}
	\vs}

%%%%%%%%%%%%%%%%%%%%%%%%%%%%%%%%%%%%%%%%%%%%%%%%%%%%%%%%%%%%%%%%%%%%%%%%%%%%%%
%% MC formatting for 1 option per line

\newcommand{\MCone}[4]
	{\phantom{x}\hspace{.07in}
	\begin{tabular}{p{5.25in}}
		{\bf A)}\, #1\\
		{\bf B)}\, #2\\
		{\bf C)}\, #3\\
		{\bf D)}\, #4
	\end{tabular}
	\vs\svs}

%%%%%%%%%%%%%%%%%%%%%%%%%%%%%%%%%%%%%%%%%%%%%%%%%%%%%%%%%%%%%%%%%%%%%%%%%%%%%%
%% MC formatting for 2 options per line

\newcommand{\MCtwo}[4]
	{\phantom{x}\hspace{.07in}
	\begin{tabular}{p{2.57in}p{2.57in}}
		{\bf A)}\, #1
		&{\bf B)}\, #2 \\*[1pc]
		{\bf C)}\, #3
		&{\bf D)}\, #4
	\end{tabular}
	\vs\svs}

%%%%%%%%%%%%%%%%%%%%%%%%%%%%%%%%%%%%%%%%%%%%%%%%%%%%%%%%%%%%%%%%%%%%%%%%%%%%%%
%% MC formatting for 4 options per line

\newcommand{\MCfour}[4]
	{\phantom{x}\hspace{.07in}
	\begin{tabular}{p{1.2in}p{1.2in}p{1.2in}p{1.2in}}
	 	{\bf A)}\, #1
		&{\bf B)}\, #2
		&{\bf C)}\, #3
		&{\bf D)}\, #4
	\end{tabular}
	\svs}

  %%%%%%%%%%%%%%%%%%%%%%%%%%%%%%%%%%%%%%%%%%%%%%%%%%%%%%%%%%%%%%%%%%%%%%%%%%%%%%%%
